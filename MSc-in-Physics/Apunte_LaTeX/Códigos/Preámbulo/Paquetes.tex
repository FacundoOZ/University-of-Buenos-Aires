
%============================================================================================================================================
% Librerías
%============================================================================================================================================

% IMPORTANTE: cambiar los package de lugar (intercambiar entre arriba y abajo) puede ocasionar problemas a la hora de compilar el documento.

% Configuración general del documento:
\usepackage[utf8]{inputenc}			% Codificación UTF-8 para caracteres especiales
\usepackage[T1]{fontenc}			% Codificación de fuente T1 (mejor manejo de acentos y símbolos europeos)
\usepackage[spanish,english]{babel}	% Idiomas: Español, Inglés
\usepackage{geometry}				% Control de márgenes y dimensiones de página
\usepackage{float}					% Control de posición exacta de figuras y tablas (opción [H])
\usepackage{pdflscape}				% Páginas en orientación apaisada dentro del PDF
\usepackage{setspace}				% Control de interlineado (simple, 1.5, doble, etc.)
\usepackage[none]{hyphenat}			% Elimina la separación de palabras por sílabas
\usepackage{titlesec}				% Personalización de secciones, capítulos y títulos
\usepackage{fancyhdr}				% Encabezados y pies de página personalizados

% Fuentes y tipografía:
\usepackage{lmodern}				% Fuente Latin Modern (mejor que la Computer Modern por defecto)
\usepackage{color}					% Soporte básico de color
\usepackage{xargs}					% Definición de comandos con argumentos opcionales múltiples
\usepackage{url}					% Permite mostrar y formatear direcciones URL correctamente
%\usepackage[skip=0.25cm]{parskip} 	% Espaciado entre párrafos en lugar de sangría

% Matemática:
\usepackage{amsmath}				% Paquete de matemática principal (alinear ecuaciones, entornos, etc.)
\usepackage{mathtools}				% Extensión de amsmath con funciones adicionales
\usepackage{amsfonts}				% Fuentes matemáticas adicionales (como \mathbb)
\usepackage{amssymb}				% Símbolos matemáticos adicionales
\usepackage{bm}						% Negrita para símbolos matemáticos
\usepackage{nccmath}				% Permite alinear ecuaciones a la izquierda o modificar el espaciado
\usepackage{esint}					% Integrales extendidas (como integrales de superficie cerrada)
\usepackage{xfp}					% Cálculos aritméticos en LaTeX (float-point)

% Tablas:
\usepackage{booktabs}				% Mejora el diseño de tablas (líneas horizontales más elegantes)
\usepackage{multirow}				% Permite combinar varias filas en tablas
\usepackage[table]{xcolor}			% Tablas con colores (permite colorear filas y columnas)
\usepackage[math]{cellspace}		% Espaciado interno en celdas de tablas (útil con expresiones matemáticas)

% Figuras, gráficos y entornos visuales:
\usepackage{graphicx}				% Inclusión y manejo de imágenes
\usepackage{caption}				% Personalización de subtítulos de figuras y tablas
\usepackage{subfig}					% Subfiguras dentro de una misma figura
\usepackage[subfigure]{tocloft}		% Complemento para subfiguras en listas de figuras
\usepackage{wrapfig}				% Inserta figuras dentro del texto (envolventes)
\usepackage{animate}				% Inserta animaciones (por ejemplo, secuencias de imágenes)

% Gráficos matemáticos y físicos:
\usepackage{pgfplots}				% Gráficos de funciones y datos con TikZ
\usepackage{circuitikz}				% Dibujos de circuitos eléctricos y diagramas físicos con TikZ
\usepackage{stackengine}			% Permite apilar objetos (texto, símbolos, etc.) unos sobre otros
\usepackage{wasysym}				% Símbolos adicionales (astronómicos, eléctricos, etc.)
\usetikzlibrary{					% Librerías adicionales para TikZ y circuitikz
	positioning,					% Control avanzado de posiciones relativas de nodos
	fit,							% Agrupar nodos en un solo recuadro
	calc,							% Cálculos de coordenadas (operaciones matemáticas con posiciones)
	angles,							% Dibujo y etiquetado de ángulos
	quotes,							% Anotaciones tipo "quotes" en diagramas
	arrows							% Control de tipos de flechas
}
\tikzset{>=latex}					% Estilo de flechas tipo LaTeX (con punta rellena)

% Código fuente y programación:
\usepackage{listings}				% Entorno para mostrar código fuente (resaltado sintáctico, etc.)

% Bibliografía:
\usepackage{biblatex}				% Manejo moderno de bibliografía (compatible con Biber/BibTeX)

% Otros:
\usepackage{xstring}				% Manipulación de cadenas de texto dentro de LaTeX

