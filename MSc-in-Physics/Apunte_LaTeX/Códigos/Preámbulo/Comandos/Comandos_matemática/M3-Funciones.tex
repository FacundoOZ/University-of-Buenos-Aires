
%============================================================================================================================================
% Comandos de matemática
%============================================================================================================================================

% Parte 3: Funciones.
%————————————————————

\setlength\defaultaddspace{0.75ex}

% Variables de funciones:
\NC{\x}{(x)}                                        % Cartesianas 1D
\NC{\y}{(y)}
\NC{\z}{(z)}
\NC{\xy}{(x,y)}                                     % Cartesianas 2D
\NC{\xz}{(x,z)}
\NC{\yz}{(y,z)}
\NC{\xyz}{(x,y,z)}                                  % Cartesianas 3D
\NCX{\prs}[3][1=1,3=n]{({#2}_{#1},\pors,{#2}_{#3})} % Función De n-Variables.

\NC{\rf}{(\ro,\phi)}      % Polares	  = (ro, phi)
\NC{\rfz}{(\ro,\phi,z)}   % Cilíndricas = (ro, phi, z)
\NC{\rtf}{(r,\tita,\phi)} % Esféricas	  = (r, tita, phi)

% Variables + tiempo:
\NC{\xt}{(x,t)}              % Cartesianas 1D
\NC{\xyt}{(x,y,t)}           % Cartesianas 2D
\NC{\xyzt}{(x,y,z,t)}        % Cartesianas 3D
\NC{\rft}{(\ro,\phi,t)}      % Polares
\NC{\rfzt}{(\ro,\phi,z,t)}   % Cilíndricas
\NC{\rtft}{(r,\tita,\phi,t)} % Esféricas

% Operaciones de funciones:
\NC{\comp}[2]{#1 \circ #2}                                                % Composición
\NC{\conv}[2]{#1 * #2}                                                    % Convolución
\NC{\rectan}[2]{\tx{R}_{\tx{T}}\left[{{#1}_{\,\left({#2}\right)}}\right]} % Recta tangente
\NC{\platan}[2]{\Pi_{\tx{T}}\left[{{#1}_{\,\left({#2}\right)}}\right]}    % Plano tangente

% Análisis:
\NC{\dom}[1]{\bb{D}_{\,\left({#1}\right)}} % Dominio
\NC{\cod}[1]{\bb{C}_{\,\left({#1}\right)}} % Codominio
\NC{\img}[1]{\bb{I}_{\,\left({#1}\right)}} % Imagen

% Funciones Trigonométricas:
\Mo{\sen}{sen}       % Seno
\Mo{\asen}{arcsen}   % Arcoseno
\Mo{\acos}{arccos}   % Arcocoseno
\Mo{\atan}{arctan}   % Arcotangente
\Mo{\acsc}{arccsc}   % Arcocosecante
\Mo{\asec}{arcsec}   % Arcosecante
\Mo{\acot}{arccot}   % Arcocotangente
\Mo{\senh}{senh}     % Seno Hiperbólico
\Mo{\csch}{csch}     % Cosecante Hiperbólica
\Mo{\sech}{sech}     % Secante Hiperbólica
\Mo{\asenh}{arcsenh} % Arcoseno Hiperbólico
\Mo{\acosh}{arccosh} % Arcocoseno Hiperbólico
\Mo{\atanh}{arctanh} % Arcotangente Hiperbólica
\Mo{\acsch}{arccsch} % Arcocosecante Hiperbólica
\Mo{\asech}{arcsech} % Arcosecante Hiperbólica
\Mo{\acoth}{arccoth} % Arcocotangente Hiperbólica

% Polinomios:
\Mo{\grad}{grad}                                       % Grado
\Mo{\coefp}{cp}                                        % Coeficiente principal
\NC{\mult}[2]{\tx{mult}\esp{-3}\left\{{#2,#1}\right\}} % Multiplicidad

\NC{\Polh}[2]{H_{#1\left({#2}\right)}}                             % Polinomios de Hermite
\NC{\PolH}[2]{H_{#1}\esp{-4}\left({\SB{0.7}{$#2$}}\right)}
\NC{\Polle}[3][]{P_{#2\left({#3}\right)}^{#1}}                     % Polinomios de Legendre
\NC{\Pollet}[2][]{P_{#2(\cos\tita)}^{#1}}                          % Polinomios de Legendre Trigonométricos
\NC{\Polla}[3][]{L_{#2\left({#3}\right)}^{#1}}                     % Polinomios de Laguerre
\NC{\PolLa}[3][]{L_{#2}^{#1}\esp{-4}\left({\SB{0.7}{$#3$}}\right)}
\NC{\Polch}[2]{T_{#1\left({#2}\right)}}                            % Polinomios de Chebyshev
\NC{\Polchs}[2]{U_{#1\left({#2}\right)}}                           % Polinomios de Chebyshev de Segunda Especie
\NC{\Polcht}[1]{T_{#1(\cos\tita)}}                                 % Polinomios de Chebyshev Trigonométricos
\NC{\Polchst}[1]{U_{#1(\cos\tita)}}                                % Polinomios de Chebyshev de Segunda Especie Trigonométricos
\NC{\Polja}[4]{P_{#1\left({#4}\right)}^{\left({#2,#3}\right)}}     % Polinomios de Jacobi

% Funciones partidas:
\NC{\crc}[2]{\ji_{#1\,\left({#2}\right)}}   % Función característica
\Mo{\sg}{sg}                                % Función signo
\NC{\heav}[1]{\titaj_{\,\left({#1}\right)}} % Función de Heaviside
\Mo{\rect}{rect}                            % Función rectángulo
\Mo{\abs}{abs}                              % Función valor absoluto
\Mo{\ramp}{ramp}                            % Función rampa
\Mo{\techo}{techo}                          % Función techo
\Mo{\piso}{piso}                            % Función piso

% Funciones antiderivadas de funciones elementales:
\Mo{\Ei}{Ei}
\Mo{\li}{li}
\Mo{\Li}{Li}
\Mo{\si}{si}
\Mo{\Si}{Si}
\Mo{\Ci}{Ci}
\Mo{\Shi}{Shi}
\Mo{\Chi}{Chi}
\Mo{\erf}{erf}

% Funciones especiales:
\Mo{\senc}{senc}                                                                    % Seno cardinal
\NC{\ArmS}[3][]{Y_{#2(\tita #1,\phi #1)}^{#3}}                                      % Armónicos esféricos
\NC{\ArmSr}[2]{Y_{#1,#2}}
\NC{\ArmSc}[3][]{Y_{#2(\tita #1,\phi #1)}^{#3 \esp{3} \SB{0.9}{*}}}
\NC{\G}[1]{\Gama_{\left({#1}\right)}} \NC{\PG}[2][]{\psi_{\left({#2}\right)}^{#1}}	% Función Gamma
\NC{\BesJ}[2]{J_{#1\esp{3}\left({#2}\right)}}                                       % Funciones de Bessel
\NC{\BesN}[2]{N_{#1\esp{3}\left({#2}\right)}}
\NC{\BesI}[2]{I_{#1\esp{3}\left({#2}\right)}}
\NC{\BesK}[2]{K_{#1\esp{3}\left({#2}\right)}}

% Distribuciones:
\NC{\dirac}[2][]{\del_{#2}^{#1}}              % Delta de Dirac
\NC{\kro}[2][]{\del_{#2}^{#1}} \Mo{\var}{var} % Delta de Kronecker

\NC{\Fs}[2]{#1 : \bb{D} \inc \bb{R}^{#2} \to \bb{R}}           % Función escalar
\NC{\Fv}[3]{#1 : \bb{D} \inc \bb{R}^{#2} \to \bb{R}^{#3}}      % Función vectorial
\NC{\Cv}[2]{\ff{#1} : \bb{D} \inc \bb{R}^{#2} \to \bb{R}^{#2}} % Campo vectorial

\NC{\Def}[5]{                                         % Definición de una función
	\begin{array}{llccl}
		& #1 : & #2 & \To & #3 \\
		& & #4 & \to & \boxed{#5}
	\end{array}
}
\NC{\Defc}[6]{                                        % Definición de una función con parámetro
	\begin{array}{llccl}
		& #1 : & #2 & \To & #3 \\
		& & #4 & \to & \boxed{#5} \tx{ , con } \boxed{#6}
	\end{array}
}

% Funciones definidas a trozos (casos):
\NC{\Ftt}[5][]{                                        % Dos casos
	\left\{\begin{array}{SlSrSr}
		$\esp{-10} #2$ & $\esp{20}$ #1 $\esp{5}$ & $#3$ \\
		$\esp{-10} #4$ & $\esp{20}$ #1 $\esp{5}$ & $#5$
	\end{array}\right. \esp{-9}
}
\NC{\Fttt}[7][]{                                       % Tres casos
	\left\{\begin{array}{SlSrSr}
		$\esp{-10} #2$ & $\esp{20}$ #1 $\esp{5}$ & $#3$ \\
		$\esp{-10} #4$ & $\esp{20}$ #1 $\esp{5}$ & $#5$ \\
		$\esp{-10} #6$ & $\esp{20}$ #1 $\esp{5}$ & $#7$
	\end{array}\right. \esp{-9}
}
\NC{\Ftttt}[9][]{                                      % Cuatro casos
	\left\{\begin{array}{SlSrSr}
		$\esp{-10} #2$ & $\esp{20}$ #1 $\esp{5}$ & $#3$ \\
		$\esp{-10} #4$ & $\esp{20}$ #1 $\esp{5}$ & $#5$ \\
		$\esp{-10} #6$ & $\esp{20}$ #1 $\esp{5}$ & $#7$ \\
		$\esp{-10} #8$ & $\esp{20}$ #1 $\esp{5}$ & $#9$
	\end{array}\right. \esp{-9}
}
\NC{\Ftn}[9][]{                                        % n-casos (3 ejemplos + ... + n-ésimo)
	\left\{\begin{array}{SlSrSr}
		$\esp{-10} #2$ & $\esp{20}$ #1 $\esp{5}$ & $#3$ \\
		$\esp{-10} #4$ & $\esp{20}$ #1 $\esp{5}$ & $#5$ \\
		$\esp{-10} #6$ & $\esp{20}$ #1 $\esp{5}$ & $#7$ \\
		\quad \porv \\
		$\esp{-10} #8$ & $\esp{20}$ #1 $\esp{5}$ & $#9$
	\end{array}\right. \esp{-9}
}

% Funciones partidas (con llave grande):
\NC{\Fp}[1]{                  % Un caso
	\left\{\begin{array}{Sl}
		$\esp{-10} #1$
	\end{array}\right. \esp{-9}
}
\NC{\Fpp}[2]{                 % Dos casos
	\left\{\begin{array}{Sl}
		$\esp{-10} #1$ \\
		$\esp{-10} #2$
	\end{array}\right. \esp{-9}
}
\NC{\Fppp}[3]{                % Tres casos
	\left\{\begin{array}{Sl}
		$\esp{-10} #1$ \\
		$\esp{-10} #2$ \\
		$\esp{-10} #3$
	\end{array}\right. \esp{-9}
}
\NC{\Fpppp}[4]{               % Cuatro casos
	\left\{\begin{array}{Sl}
		$\esp{-10} #1$ \\
		$\esp{-10} #2$ \\
		$\esp{-10} #3$ \\
		$\esp{-10} #4$
	\end{array}\right. \esp{-9}
}
\NC{\Fppppp}[5]{              % Cinco casos
	\left\{\begin{array}{Sl}
		$\esp{-10} #1$ \\
		$\esp{-10} #2$ \\
		$\esp{-10} #3$ \\
		$\esp{-10} #4$ \\
		$\esp{-10} #5$
	\end{array}\right. \esp{-9}
}
\NC{\Fpppppp}[6]{             % Seis casos
	\left\{\begin{array}{Sl}
		$\esp{-10} #1$ \\
		$\esp{-10} #2$ \\
		$\esp{-10} #3$ \\
		$\esp{-10} #4$ \\
		$\esp{-10} #5$ \\
		$\esp{-10} #6$
	\end{array}\right. \esp{-9}
}
\NC{\Fppppppp}[7]{            % Siete casos
	\left\{\begin{array}{Sl}
		$\esp{-10} #1$ \\
		$\esp{-10} #2$ \\
		$\esp{-10} #3$ \\
		$\esp{-10} #4$ \\
		$\esp{-10} #5$ \\
		$\esp{-10} #6$ \\
		$\esp{-10} #7$
	\end{array}\right. \esp{-9}
}
\NC{\Fpppppppp}[8]{           % Ocho casos
	\left\{\begin{array}{Sl}
		$\esp{-10} #1$ \\
		$\esp{-10} #2$ \\
		$\esp{-10} #3$ \\
		$\esp{-10} #4$ \\
		$\esp{-10} #5$ \\
		$\esp{-10} #6$ \\
		$\esp{-10} #7$ \\
		$\esp{-10} #8$
	\end{array}\right. \esp{-9}
}
\NC{\Fpn}[5][\quad]{          % n-casos (3 ejemplos + ... + n-ésimo)
	\left\{\begin{array}{Sl}
		$\esp{-10} #2$ \\
		$\esp{-10} #3$ \\
		$\esp{-10} #4$ \\
		$#1$ \porv \\
		$\esp{-10} #5$
	\end{array}\right. \esp{-9}
}
\NC{\Fppn}[5]{                % n-casos (4 ejemplos + ... + n-ésimo)
	\left\{\begin{array}{Sl}
		$\esp{-10} #1$ \\
		$\esp{-10} #2$ \\
		$\esp{-10} #3$ \\
		$\esp{-10} #4$ \\
		\quad \porv \\
		$\esp{-10} #5$
	\end{array}\right. \esp{-9}
}

%————————————————————————————————————————————————————————————————————————————————————————————————————————————————————————————————————————————