
%============================================================================================================================================
% Comandos de matemática
%============================================================================================================================================

% Parte 1: Símbolos matemáticos.
%-------------------------------

% Agrupación en paréntesis, corchetes, llaves, etc.:
\NC{\pr}   [2][]{{\!\left({#2}\right)}^{#1} \esp{-2}}					% Paréntesis
\NC{\pra}  [2][]{{\!\left\langle{#2}\right\rangle}^{#1} \esp{-2}}		% Brackets
\NC{\cor}  [2][]{{\!\left[{#2}\right]}^{#1} \esp{-2}}					% Corchetes
\NC{\lla}  [2][]{{\!\left\{{#2}\right\}}^{#1} \esp{-2}}					% Llaves
\let\Mod\mod
\RC{\mod}  [2][]{\left|{#2}\right|^{#1}}								% Módulo
\NCX{\norm}[3][1=,3=]{\Mc{{\left|\left|{#2}\right|\right|}_{#1}^{#3}}}	% Norma
\NC{\llar} [2][]{\overbrace{#2}^{#1}}									% Llave Arriba
\NC{\llab} [2][]{\underbrace{#2}_{#1}}									% Llave Abajo

\NC{\lc}{\!\left\lceil}													% Techo
\NC{\rc}{\right\rceil\!}
\NC{\lfl}{\!\left\lfloor}												% Piso
\NC{\rfl}{\right\rfloor\!}
\NC{\ldot}{\left.}														% Puntos
\NC{\rdot}{\right.}

\NC{\esp}[1]{\mkern #1 mu}												% Espacios (PRECAUCIÓN: este comando siempre debe estar en MATHMODE)
\NC{\quadl}{\esp{-18}}													% Quad left, es como quad pero hacia izquierda.

% Desigualdades:
\NC{\dis}   {\neq}							% Distinto
\NC{\apx}   {\approx} 						% Aproximadamente
\NC{\apxig} {\simeq} 						% Aproximadamente igual
\NC{\eqv}   {\equiv} 						% Equivalente
\NC{\eqvl}  {\leftrightsquigarrow}			% Equivalente
\NC{\neqv}  {\not\equiv}					% Equivalente
\NC{\mig}   {\geqslant} 					% Mayor o igual
\NC{\nig}   {\leqslant} 					% Menor o igual
\NC{\apxmig}{\gtrsim} 						% Mayor o aproximadamente igual
\NC{\apxnig}{\lesssim} 						% Menor o aproximadamente igual
\NC{\mm}    {\gg} 							% Mucho mayor
\NC{\nn}    {\ll} 							% Mucho menor
\NC{\mn}    {\gtrless} 						% Mayor menor
\NC{\nm}    {\lessgtr}						% Menor mayor
\NC{\mign}  {\gtreqless} 					% Mayor igual menor
\NC{\txsim}[2][]{\stackrel{\tx{#1}}{#2}}	% Símbolo con texto

% Acentos:
\NC{\vm}[1]{\left\langle #1 \right\rangle}	% Valor medio
\NC{\tc}[1]{{#1}^{\dagger}}					% Traspuesto conjugado
\NC{\monio}[1]{\tilde{#1}}					% Moño
