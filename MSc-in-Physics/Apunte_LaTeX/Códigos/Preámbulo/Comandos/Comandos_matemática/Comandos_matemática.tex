
%============================================================================================================================================
% Comandos de matemática
%============================================================================================================================================

% Estructura:
\NC{\Mo}{\DeclareMathOperator}
\NC{\Ms}{\DeclareMathSymbol}
\NC{\Op}{\operatorname}
\NC{\Mc}{\ensuremath}
\pgfplotsset{compat = 1.18, width = 10cm} % Gráficos matemáticos
\allowdisplaybreaks[4]                    % Permite que los environments aligns, equation,... continuen en la página siguiente

% La ruta de acceso debe estar completa, ya que otros archivos fuera de BOOK utilizan Comandos Matemáticos:

%============================================================================================================================================
% Comandos de matemática
%============================================================================================================================================

% Parte 0: Lógica proposicional.
%———————————————————————————————

% Símbolos lógicos:
\NC{\no}  {\lnot}                      % No
\NC{\Y}   {\wedge}                     % Y
\RC{\O}   {\vee}                       % O (no excluyente)
\NC{\Oex} {\veebar}                    % O excluyente
\NC{\ex}  {\exists \esp{9}}            % Existe
\NC{\exu} {\exists! \esp{9}}           % Existe un único
\NC{\nex} {\nexists \esp{9}}           % No existe
\NC{\ptd} {\esp{8} \forall \esp{8}}    % Para todo
\NC{\plt} {\esp{6} \therefore \esp{6}} % Por lo tanto
\NC{\porq}{\because}                   % Porque
\NC{\prop}{\propto}                    % Proporcional
\NC{\QED} {\blacksquare}               % Queda demostrado
\NC{\pors}{\dots}                      % Puntos suspensivos
\NC{\porv}{\vdots}                     % Puntos verticales
\NC{\porh}{\cdots}                     % Puntos centrados
\NC{\pord}{\ddots}                     % Puntos diagonales
\NC{\grs} {^{\circ}}                   % Grados
\NC{\rel} [1][]{\esp{2}             \cal{R}_{#1}\esp{3}} % Relacionado
\NC{\nrel}[1][]{\esp{-1}\not\esp{-6}\cal{R}_{#1}\esp{3}} % No relacionado

% Conjuntos numéricos:
\NC{\nin}  {\notin}           % No pertenece
\NC{\inc}  {\subseteq}        % Incluído
\NC{\ninc} {\not\subseteq}    % No incluído
\NC{\vacio}{\varnothing}      % Conjunto vacío
\NC{\compl}[1]{{#1}^{\tx{c}}} % Complemento
\NC{\union}{\cup}             % Unión
\NC{\intsc}{\cap}             % Intersección
\NC{\difs} {\vartriangle}     % Diferencia simétrica
\NC{\PC}   {\Por}             % Producto cartesiano
\RC{\inf}  {\infty}           % Infinito
\NC*{\card}[1]{                                              % Cardinal
	\begin{tikzpicture}
		\pgfmathsetlengthmacro\myWidth{.8*width("=")}            % Ancho = 80% del signo igual
		\pgfmathsetlengthmacro\myHeight{height("H")}             % Altura = altura de la letra H mayúscula
		\pgfmathsetlengthmacro\mySepY{.3333*\myWidth}            % Espacio entre líneas horizontales
		\pgfmathsetlengthmacro\mySideBearing{.1*\myWidth}        % Soporte lateral
		\def\myAngle{70}                                         % Ángulo de las líneas verticales inclinadas
		\pgfmathsetlengthmacro\mySepX{\mySepY/sin(\myAngle)}     % Calcula la separación de las líneas verticales en dirección horizontal
		\pgfmathsetlengthmacro\mySlantX{\myHeight/tan(\myAngle)} % Calcula el ancho de una línea inclinada
		\draw[line cap=round]
			(0, {(\myHeight - \mySepY)/2}) -- ++(\myWidth, 0)
			(0, {(\myHeight + \mySepY)/2}) -- ++(\myWidth, 0)
			({(\myWidth - \mySepX - \mySlantX)/2}, 0)
			-- ({(\myWidth - \mySepX + \mySlantX)/2}, \myHeight)
			({(\myWidth + \mySepX - \mySlantX)/2}, 0)
			-- ({(\myWidth + \mySepX + \mySlantX)/2}, \myHeight);
		\useasboundingbox
			(-\mySideBearing, 0)
			(\myWidth + \mySideBearing, \myHeight)
		;
	\end{tikzpicture}
	{#1}
}

% Flechas:
\NC{\To}    {\longrightarrow}     % Hacia
\NC{\ent}   {\Rightarrow}         % Entonces
\NC{\Ent}   {\Longrightarrow}     % Entonces largo
\NC{\sii}   {\Leftrightarrow}     % Si y solo si
\NC{\Sii}   {\Longleftrightarrow} % Si y solo si largo
\NC{\Tiende}{\xrightarrow}        % Tiende
\NC{\Sneg}  {\circlearrowright}   % Sentido negativo (horario)
\NC{\Spos}  {\circlearrowleft}    % Sentido positivo (anti-horario)

%————————————————————————————————————————————————————————————————————————————————————————————————————————————————————————————————————————————

%============================================================================================================================================
% Comandos de matemática
%============================================================================================================================================

% Parte 1: Símbolos matemáticos.
%-------------------------------

% Agrupación en paréntesis, corchetes, llaves, etc.:
\NC{\pr}   [2][]{{\!\left({#2}\right)}^{#1} \esp{-2}}					% Paréntesis
\NC{\pra}  [2][]{{\!\left\langle{#2}\right\rangle}^{#1} \esp{-2}}		% Brackets
\NC{\cor}  [2][]{{\!\left[{#2}\right]}^{#1} \esp{-2}}					% Corchetes
\NC{\lla}  [2][]{{\!\left\{{#2}\right\}}^{#1} \esp{-2}}					% Llaves
\let\Mod\mod
\RC{\mod}  [2][]{\left|{#2}\right|^{#1}}								% Módulo
\NCX{\norm}[3][1=,3=]{\Mc{{\left|\left|{#2}\right|\right|}_{#1}^{#3}}}	% Norma
\NC{\llar} [2][]{\overbrace{#2}^{#1}}									% Llave Arriba
\NC{\llab} [2][]{\underbrace{#2}_{#1}}									% Llave Abajo

\NC{\lc}{\!\left\lceil}													% Techo
\NC{\rc}{\right\rceil\!}
\NC{\lfl}{\!\left\lfloor}												% Piso
\NC{\rfl}{\right\rfloor\!}
\NC{\ldot}{\left.}														% Puntos
\NC{\rdot}{\right.}

\NC{\esp}[1]{\mkern #1 mu}												% Espacios (PRECAUCIÓN: este comando siempre debe estar en MATHMODE)
\NC{\quadl}{\esp{-18}}													% Quad left, es como quad pero hacia izquierda.

% Desigualdades:
\NC{\dis}   {\neq}							% Distinto
\NC{\apx}   {\approx} 						% Aproximadamente
\NC{\apxig} {\simeq} 						% Aproximadamente igual
\NC{\eqv}   {\equiv} 						% Equivalente
\NC{\eqvl}  {\leftrightsquigarrow}			% Equivalente
\NC{\neqv}  {\not\equiv}					% Equivalente
\NC{\mig}   {\geqslant} 					% Mayor o igual
\NC{\nig}   {\leqslant} 					% Menor o igual
\NC{\apxmig}{\gtrsim} 						% Mayor o aproximadamente igual
\NC{\apxnig}{\lesssim} 						% Menor o aproximadamente igual
\NC{\mm}    {\gg} 							% Mucho mayor
\NC{\nn}    {\ll} 							% Mucho menor
\NC{\mn}    {\gtrless} 						% Mayor menor
\NC{\nm}    {\lessgtr}						% Menor mayor
\NC{\mign}  {\gtreqless} 					% Mayor igual menor
\NC{\txsim}[2][]{\stackrel{\tx{#1}}{#2}}	% Símbolo con texto

% Acentos:
\NC{\vm}[1]{\left\langle #1 \right\rangle}	% Valor medio
\NC{\tc}[1]{{#1}^{\dagger}}					% Traspuesto conjugado
\NC{\monio}[1]{\tilde{#1}}					% Moño


%============================================================================================================================================
% Comandos de matemática
%============================================================================================================================================

% Parte 2: Vectores.
%-------------------

% Vectores:
\DeclareSymbolFont{mathbf}{OT1}{\familydefault}{\bfdefault}{n}
\DeclareSymbolFontAlphabet{\ff}{mathbf}
\Ms{\boldimath}{\mathord}{mathbf}{"10}
\Ms{\boldjmath}{\mathord}{mathbf}{"11}
\ExplSyntaxOn
	\NDC{\vc}{m}{											% Vector en negrita:
		\token_if_cs:NTF #1 {\bm{#1}}{						% si es una letra griega, aplicamos boldsymbol (\bm)
			\str_case:nnF{#1}{
				{i}{\ff{\boldimath}}						% si es i ó j, le quitamos el punto y lo hacemos bold,
				{j}{\ff{\boldjmath}}
			}
			{\ff{#1}}										% si no, es simplemente mathbf (\ff)
		}
	}
	\NDC{\vco}{m}{											% Vector en negrita, medido desde un centro de momentos (o):
		\token_if_cs:NTF #1 {{\bm{#1}}^{\SB{0.5}{(o)}}}{	% Ídem que vector común, pero le agrego (o) de exponente y achicado en 0.5
			\str_case:nnF{#1}{
				{i}{\ff{\boldimath}}
				{j}{\ff{\boldjmath}}
			}
			{{\ff{#1}}^{\SB{0.5}{(o)}}}
		}
	}
	\NDC{\vcm}{m}{											% Vector en negrita, medido desde el centro de masas (cm):
		\token_if_cs:NTF #1 {{\bm{#1}}^{\SB{0.5}{(\tx{cm})}}}{	% Ídem que \vco, pero con cm.
			\str_case:nnF{#1}{
				{i}{\ff{\boldimath}}
				{j}{\ff{\boldjmath}}
			}
			{{\ff{#1}}^{\SB{0.5}{(\tx{cm})}}}
		}
	}
	\NDC{\ver}{m}{											% Versor:
		\token_if_cs:NTF #1 {\bm{\hat{#1}}}{				% Es igual que \vc{}, pero lleva sombrero (\hat)
			\str_case:nnF{#1}{
				{i}{\ff{\hat{\boldimath}}}
				{j}{\ff{\hat{\boldjmath}}}
			}
			{\ff{\hat{#1}}}
		}
	}
\ExplSyntaxOff

% Tensor:
\NCX{\ten}[3][1=,3=]{\cal{#2}_{#1}^{#3}}

% Otro: vector en notación vieja
\RC{\Vec}{\overrightarrow}
\NC{\Veco}[1]{\Vec{{#1}^{\esp{-6}\SB{0.5}{(o)}}}}		% Vector medido desde un centro de momentos (o)
\NC{\Vecm}[1]{\Vec{{#1}^{\esp{-6}\SB{0.5}{(\tx{cm})}}}} % Vector medido desde el centro de masas (cm)

% Operaciones entre vectores:
\NC{\por}{\cdot} 											% Operador producto interno
\NC{\Por}{\times} 											% Operador producto vectorial
\NC{\Tor}{\otimes} 											% Operador producto tensorial
\NC{\PI}[2]{#1 \por #2}										% Producto interno   entre elementos
\NC{\PV}[2]{#1 \Por #2} 									% Producto vectorial entre elementos
\NC{\PT}[2]{#1 \Tor #2} 									% Producto tensorial entre elementos
\NC{\SPI}[2]{\left\langle{#1 \esp{3} , #2}\right\rangle}	% Espacio con producto interno
\NC{\compnt}[2]{\Op{comp}_{#1} \left(#2\right)}				% Componente
\NC{\proy}[2]{\Op{proy}_{#1} \left(#2\right)}				% Proyección ortogonal
\NC{\levi}[2][]{\epsj_{#2}^{#1}}							% Símbolo de Levi-Civita
\NC{\PVr}[6]{\left\langle{#2 #6 - #3 #5,#3 #4 - #1 #6,#1 #5 - #2 #4}\right\rangle}			% Producto vectorial resuelto
\NC{\PVc}[6]{\left\langle{#1,#2,#3}\right\rangle \Por \left\langle{#4,#5,#6}\right\rangle}	% Producto vectorial en componentes

% Operadores vectoriales:
\NC{\gr}[2][]{\nabla_{#1} {#2}}				% Gradiente
\RC{\div}[2][]{\nabla_{#1} \por \vc{#2}}	% Divergencia
\NC{\rot}[2][]{\nabla_{#1} \Por \vc{#2}}	% Rotacional
\NC{\lap}[2][]{\nabla_{#1}^2 {#2}}			% Laplaciano
\NC{\lapv}[2][]{\nabla_{#1}^2 \vc{#2}}		% Laplaciano vectorial
\NC{\blap}[1]{\nabla^4 {#1}}				% Bilaplaciano
\NC{\dal}[2][]{\Box_{#1} {#2}}				% D'Alambertiano
\NC{\dalv}[2][]{\Box_{#1} \vc{#2}}			% D'Alambertiano vectorial
\NC{\cnv}[2]{(\PI{\vc{#1}}{\nabla}) #2}		% Derivada Convectiva
\NC{\Rotr}[3]{								% Rotacional resuelto
	\left\langle{
		\pd{{#3}_{\xyz}}{y} - \pd{{#2}_{\xyz}}{z},
		\pd{{#1}_{\xyz}}{z} - \pd{{#3}_{\xyz}}{x},
		\pd{{#2}_{\xyz}}{x} - \pd{{#1}_{\xyz}}{y}
	}\right\rangle
}


%============================================================================================================================================
% Comandos de matemática
%============================================================================================================================================

% Parte 3: Funciones.
%————————————————————

\setlength\defaultaddspace{0.75ex}

% Variables de funciones:
\NC{\x}{(x)}                                        % Cartesianas 1D
\NC{\y}{(y)}
\NC{\z}{(z)}
\NC{\xy}{(x,y)}                                     % Cartesianas 2D
\NC{\xz}{(x,z)}
\NC{\yz}{(y,z)}
\NC{\xyz}{(x,y,z)}                                  % Cartesianas 3D
\NCX{\prs}[3][1=1,3=n]{({#2}_{#1},\pors,{#2}_{#3})} % Función De n-Variables.

\NC{\rf}{(\ro,\phi)}      % Polares	  = (ro, phi)
\NC{\rfz}{(\ro,\phi,z)}   % Cilíndricas = (ro, phi, z)
\NC{\rtf}{(r,\tita,\phi)} % Esféricas	  = (r, tita, phi)

% Variables + tiempo:
\NC{\xt}{(x,t)}              % Cartesianas 1D
\NC{\xyt}{(x,y,t)}           % Cartesianas 2D
\NC{\xyzt}{(x,y,z,t)}        % Cartesianas 3D
\NC{\rft}{(\ro,\phi,t)}      % Polares
\NC{\rfzt}{(\ro,\phi,z,t)}   % Cilíndricas
\NC{\rtft}{(r,\tita,\phi,t)} % Esféricas

% Operaciones de funciones:
\NC{\comp}[2]{#1 \circ #2}                                                % Composición
\NC{\conv}[2]{#1 * #2}                                                    % Convolución
\NC{\rectan}[2]{\tx{R}_{\tx{T}}\left[{{#1}_{\,\left({#2}\right)}}\right]} % Recta tangente
\NC{\platan}[2]{\Pi_{\tx{T}}\left[{{#1}_{\,\left({#2}\right)}}\right]}    % Plano tangente

% Análisis:
\NC{\dom}[1]{\bb{D}_{\,\left({#1}\right)}} % Dominio
\NC{\cod}[1]{\bb{C}_{\,\left({#1}\right)}} % Codominio
\NC{\img}[1]{\bb{I}_{\,\left({#1}\right)}} % Imagen

% Funciones Trigonométricas:
\Mo{\sen}{sen}       % Seno
\Mo{\asen}{arcsen}   % Arcoseno
\Mo{\acos}{arccos}   % Arcocoseno
\Mo{\atan}{arctan}   % Arcotangente
\Mo{\acsc}{arccsc}   % Arcocosecante
\Mo{\asec}{arcsec}   % Arcosecante
\Mo{\acot}{arccot}   % Arcocotangente
\Mo{\senh}{senh}     % Seno Hiperbólico
\Mo{\csch}{csch}     % Cosecante Hiperbólica
\Mo{\sech}{sech}     % Secante Hiperbólica
\Mo{\asenh}{arcsenh} % Arcoseno Hiperbólico
\Mo{\acosh}{arccosh} % Arcocoseno Hiperbólico
\Mo{\atanh}{arctanh} % Arcotangente Hiperbólica
\Mo{\acsch}{arccsch} % Arcocosecante Hiperbólica
\Mo{\asech}{arcsech} % Arcosecante Hiperbólica
\Mo{\acoth}{arccoth} % Arcocotangente Hiperbólica

% Polinomios:
\Mo{\grad}{grad}                                       % Grado
\Mo{\coefp}{cp}                                        % Coeficiente principal
\NC{\mult}[2]{\tx{mult}\esp{-3}\left\{{#2,#1}\right\}} % Multiplicidad

\NC{\Polh}[2]{H_{#1\left({#2}\right)}}                             % Polinomios de Hermite
\NC{\PolH}[2]{H_{#1}\esp{-4}\left({\SB{0.7}{$#2$}}\right)}
\NC{\Polle}[3][]{P_{#2\left({#3}\right)}^{#1}}                     % Polinomios de Legendre
\NC{\Pollet}[2][]{P_{#2(\cos\tita)}^{#1}}                          % Polinomios de Legendre Trigonométricos
\NC{\Polla}[3][]{L_{#2\left({#3}\right)}^{#1}}                     % Polinomios de Laguerre
\NC{\PolLa}[3][]{L_{#2}^{#1}\esp{-4}\left({\SB{0.7}{$#3$}}\right)}
\NC{\Polch}[2]{T_{#1\left({#2}\right)}}                            % Polinomios de Chebyshev
\NC{\Polchs}[2]{U_{#1\left({#2}\right)}}                           % Polinomios de Chebyshev de Segunda Especie
\NC{\Polcht}[1]{T_{#1(\cos\tita)}}                                 % Polinomios de Chebyshev Trigonométricos
\NC{\Polchst}[1]{U_{#1(\cos\tita)}}                                % Polinomios de Chebyshev de Segunda Especie Trigonométricos
\NC{\Polja}[4]{P_{#1\left({#4}\right)}^{\left({#2,#3}\right)}}     % Polinomios de Jacobi

% Funciones partidas:
\NC{\crc}[2]{\ji_{#1\,\left({#2}\right)}}   % Función característica
\Mo{\sg}{sg}                                % Función signo
\NC{\heav}[1]{\titaj_{\,\left({#1}\right)}} % Función de Heaviside
\Mo{\rect}{rect}                            % Función rectángulo
\Mo{\abs}{abs}                              % Función valor absoluto
\Mo{\ramp}{ramp}                            % Función rampa
\Mo{\techo}{techo}                          % Función techo
\Mo{\piso}{piso}                            % Función piso

% Funciones antiderivadas de funciones elementales:
\Mo{\Ei}{Ei}
\Mo{\li}{li}
\Mo{\Li}{Li}
\Mo{\si}{si}
\Mo{\Si}{Si}
\Mo{\Ci}{Ci}
\Mo{\Shi}{Shi}
\Mo{\Chi}{Chi}
\Mo{\erf}{erf}

% Funciones especiales:
\Mo{\senc}{senc}                                                                    % Seno cardinal
\NC{\ArmS}[3][]{Y_{#2(\tita #1,\phi #1)}^{#3}}                                      % Armónicos esféricos
\NC{\ArmSr}[2]{Y_{#1,#2}}
\NC{\ArmSc}[3][]{Y_{#2(\tita #1,\phi #1)}^{#3 \esp{3} \SB{0.9}{*}}}
\NC{\G}[1]{\Gama_{\left({#1}\right)}} \NC{\PG}[2][]{\psi_{\left({#2}\right)}^{#1}}	% Función Gamma
\NC{\BesJ}[2]{J_{#1\esp{3}\left({#2}\right)}}                                       % Funciones de Bessel
\NC{\BesN}[2]{N_{#1\esp{3}\left({#2}\right)}}
\NC{\BesI}[2]{I_{#1\esp{3}\left({#2}\right)}}
\NC{\BesK}[2]{K_{#1\esp{3}\left({#2}\right)}}

% Distribuciones:
\NC{\dirac}[2][]{\del_{#2}^{#1}}              % Delta de Dirac
\NC{\kro}[2][]{\del_{#2}^{#1}} \Mo{\var}{var} % Delta de Kronecker

\NC{\Fs}[2]{#1 : \bb{D} \inc \bb{R}^{#2} \to \bb{R}}           % Función escalar
\NC{\Fv}[3]{#1 : \bb{D} \inc \bb{R}^{#2} \to \bb{R}^{#3}}      % Función vectorial
\NC{\Cv}[2]{\ff{#1} : \bb{D} \inc \bb{R}^{#2} \to \bb{R}^{#2}} % Campo vectorial

\NC{\Def}[5]{                                         % Definición de una función
	\begin{array}{llccl}
		& #1 : & #2 & \To & #3 \\
		& & #4 & \to & \boxed{#5}
	\end{array}
}
\NC{\Defc}[6]{                                        % Definición de una función con parámetro
	\begin{array}{llccl}
		& #1 : & #2 & \To & #3 \\
		& & #4 & \to & \boxed{#5} \tx{ , con } \boxed{#6}
	\end{array}
}

% Funciones definidas a trozos (casos):
\NC{\Ftt}[5][]{                                        % Dos casos
	\left\{\begin{array}{SlSrSr}
		$\esp{-10} #2$ & $\esp{20}$ #1 $\esp{5}$ & $#3$ \\
		$\esp{-10} #4$ & $\esp{20}$ #1 $\esp{5}$ & $#5$
	\end{array}\right. \esp{-9}
}
\NC{\Fttt}[7][]{                                       % Tres casos
	\left\{\begin{array}{SlSrSr}
		$\esp{-10} #2$ & $\esp{20}$ #1 $\esp{5}$ & $#3$ \\
		$\esp{-10} #4$ & $\esp{20}$ #1 $\esp{5}$ & $#5$ \\
		$\esp{-10} #6$ & $\esp{20}$ #1 $\esp{5}$ & $#7$
	\end{array}\right. \esp{-9}
}
\NC{\Ftttt}[9][]{                                      % Cuatro casos
	\left\{\begin{array}{SlSrSr}
		$\esp{-10} #2$ & $\esp{20}$ #1 $\esp{5}$ & $#3$ \\
		$\esp{-10} #4$ & $\esp{20}$ #1 $\esp{5}$ & $#5$ \\
		$\esp{-10} #6$ & $\esp{20}$ #1 $\esp{5}$ & $#7$ \\
		$\esp{-10} #8$ & $\esp{20}$ #1 $\esp{5}$ & $#9$
	\end{array}\right. \esp{-9}
}
\NC{\Ftn}[9][]{                                        % n-casos (3 ejemplos + ... + n-ésimo)
	\left\{\begin{array}{SlSrSr}
		$\esp{-10} #2$ & $\esp{20}$ #1 $\esp{5}$ & $#3$ \\
		$\esp{-10} #4$ & $\esp{20}$ #1 $\esp{5}$ & $#5$ \\
		$\esp{-10} #6$ & $\esp{20}$ #1 $\esp{5}$ & $#7$ \\
		\quad \porv \\
		$\esp{-10} #8$ & $\esp{20}$ #1 $\esp{5}$ & $#9$
	\end{array}\right. \esp{-9}
}

% Funciones partidas (con llave grande):
\NC{\Fp}[1]{                  % Un caso
	\left\{\begin{array}{Sl}
		$\esp{-10} #1$
	\end{array}\right. \esp{-9}
}
\NC{\Fpp}[2]{                 % Dos casos
	\left\{\begin{array}{Sl}
		$\esp{-10} #1$ \\
		$\esp{-10} #2$
	\end{array}\right. \esp{-9}
}
\NC{\Fppp}[3]{                % Tres casos
	\left\{\begin{array}{Sl}
		$\esp{-10} #1$ \\
		$\esp{-10} #2$ \\
		$\esp{-10} #3$
	\end{array}\right. \esp{-9}
}
\NC{\Fpppp}[4]{               % Cuatro casos
	\left\{\begin{array}{Sl}
		$\esp{-10} #1$ \\
		$\esp{-10} #2$ \\
		$\esp{-10} #3$ \\
		$\esp{-10} #4$
	\end{array}\right. \esp{-9}
}
\NC{\Fppppp}[5]{              % Cinco casos
	\left\{\begin{array}{Sl}
		$\esp{-10} #1$ \\
		$\esp{-10} #2$ \\
		$\esp{-10} #3$ \\
		$\esp{-10} #4$ \\
		$\esp{-10} #5$
	\end{array}\right. \esp{-9}
}
\NC{\Fpppppp}[6]{             % Seis casos
	\left\{\begin{array}{Sl}
		$\esp{-10} #1$ \\
		$\esp{-10} #2$ \\
		$\esp{-10} #3$ \\
		$\esp{-10} #4$ \\
		$\esp{-10} #5$ \\
		$\esp{-10} #6$
	\end{array}\right. \esp{-9}
}
\NC{\Fppppppp}[7]{            % Siete casos
	\left\{\begin{array}{Sl}
		$\esp{-10} #1$ \\
		$\esp{-10} #2$ \\
		$\esp{-10} #3$ \\
		$\esp{-10} #4$ \\
		$\esp{-10} #5$ \\
		$\esp{-10} #6$ \\
		$\esp{-10} #7$
	\end{array}\right. \esp{-9}
}
\NC{\Fpppppppp}[8]{           % Ocho casos
	\left\{\begin{array}{Sl}
		$\esp{-10} #1$ \\
		$\esp{-10} #2$ \\
		$\esp{-10} #3$ \\
		$\esp{-10} #4$ \\
		$\esp{-10} #5$ \\
		$\esp{-10} #6$ \\
		$\esp{-10} #7$ \\
		$\esp{-10} #8$
	\end{array}\right. \esp{-9}
}
\NC{\Fpn}[5][\quad]{          % n-casos (3 ejemplos + ... + n-ésimo)
	\left\{\begin{array}{Sl}
		$\esp{-10} #2$ \\
		$\esp{-10} #3$ \\
		$\esp{-10} #4$ \\
		$#1$ \porv \\
		$\esp{-10} #5$
	\end{array}\right. \esp{-9}
}
\NC{\Fppn}[5]{                % n-casos (4 ejemplos + ... + n-ésimo)
	\left\{\begin{array}{Sl}
		$\esp{-10} #1$ \\
		$\esp{-10} #2$ \\
		$\esp{-10} #3$ \\
		$\esp{-10} #4$ \\
		\quad \porv \\
		$\esp{-10} #5$
	\end{array}\right. \esp{-9}
}

%————————————————————————————————————————————————————————————————————————————————————————————————————————————————————————————————————————————

%============================================================================================================================================
% Comandos de matemática
%============================================================================================================================================

% Parte 4: Límites.
%——————————————————

\RC{\lim}  [3]{\Lim\limits_{#1\to#2} #3}                                             % Límite
\NC{\llim} [5]{\Lim\limits_{#3\to#4} \Lim\limits_{#1\to#2} #5}                       % Límite doble
\NC{\lllim}[7]{\Lim\limits_{#5\to#6} \Lim\limits_{#3\to#4} \Lim\limits_{#1\to#2} #7} % Límite triple

% Límites compactos:
\NC{\liim} [5]{\Lim\limits_{\left({#1,#2   }\right)\,\to\,\left({#3,#4   }\right)} #5}                                     % Límite doble
\NC{\liiim}[7]{\Lim\limits_{\left({#1,#2,#3}\right)\,\to\,\left({#4,#5,#6}\right)} #7}                                     % Límite triple
\NC{\limite}[4][]{\Mc{\Lim\limits_{\left({{#2}_1,\pors,{#2}_{#1}}\right)\,\to\,\left({{#3}_1,\pors,{#3}_{#1}}\right)} #4}} % Límite n-ésimo

\NC{\lims}[3]{\Lims\limits_{#1\to#2} #3}              % Límite superior
\NC{\limi}[3]{\Limi\limits_{#1\to#2} #3}              % Límite inferior
\RC{\max} [2][]{\Max\limits_{#1} \left\{{#2}\right\}} % Máximo
\RC{\min} [2][]{\Min\limits_{#1} \left\{{#2}\right\}} % Mínimo
\RC{\sup} [2][]{\Sup\limits_{#1} \left\{{#2}\right\}} % Supremo
\NC{\infm}[2][]{\Infm\limits_{#1}\left\{{#2}\right\}} % Ínfimo

% Operadores matemáticos auxiliares:
\Mo*{\Lim} {\tx{lím}}
\Mo*{\Lims}{\tx{lím}\,\tx{sup}}
\Mo*{\Limi}{\tx{lím}\,\tx{inf}}
\Mo*{\Max} {\tx{máx}}
\Mo*{\Min} {\tx{mín}}
\Mo*{\Sup} {sup}
\Mo*{\Infm}{\tx{ínf}}

%————————————————————————————————————————————————————————————————————————————————————————————————————————————————————————————————————————————

%============================================================================================================================================
% Comandos de matemática
%============================================================================================================================================

% Parte 5: Operadores grandes.
%-----------------------------

\RC{\S}  [3]{\sum\limits_{#1}^{#2} #3}												% Sumatoria
\RC{\SS} [5]{\sum\limits_{#3}^{#4} \sum\limits_{#1}^{#2} #5}						% Suma doble
\NC{\SSS}[7]{\sum\limits_{#5}^{#6} \sum\limits_{#3}^{#4} \sum\limits_{#1}^{#2} #7}	% Suma triple

\RC{\P}  [3]{\prod\limits_{#1}^{#2} #3}													% Productoria
\NC{\PP} [5]{\prod\limits_{#3}^{#4} \prod\limits_{#1}^{#2} #5}							% Producto doble
\NC{\PPP}[7]{\prod\limits_{#5}^{#6} \prod\limits_{#3}^{#4} \prod\limits_{#1}^{#2} #7}	% Producto triple

\NC{\U}  [3]{\bigcup\limits_{#1}^{#2} #3}													% Unión
\NC{\UU} [5]{\bigcup\limits_{#3}^{#4} \bigcup\limits_{#1}^{#2} #5}							% Unión doble
\NC{\UUU}[7]{\bigcup\limits_{#5}^{#6} \bigcup\limits_{#3}^{#4} \bigcup\limits_{#1}^{#2} #7}	% Unión triple

\NC{\I}  [3]{\bigcap\limits_{#1}^{#2} #3}													% Intersección
\NC{\II} [5]{\bigcap\limits_{#3}^{#4} \bigcap\limits_{#1}^{#2} #5}							% Intersección doble
\NC{\III}[7]{\bigcap\limits_{#5}^{#6} \bigcap\limits_{#3}^{#4} \bigcap\limits_{#1}^{#2} #7}	% Intersección triple

\NC{\SD}[3]{\bigoplus\limits_{#1}^{#2} #3}													% Suma directa

%============================================================================================================================================
% Comandos de matemática
%============================================================================================================================================

% Parte 6: Derivadas.
%————————————————————

% Signos De Derivada:
\RC{\d}  [1][]     {\tx{d}_{#1}}        % Derivada
\NCX{\p} [2][1=,2=]{\partial_{#1}^{#2}} % Derivada parcial
\NC{\D}  [1][t]    {\tx{D}_{#1}}        % Derivada material
\NC{\ddir}[2]      {\tx{D}_{#1} #2}     % Derivada direccional

\NC{\dv} [3][]{\Mc{\fr{\d^{#1} #2}      {\d #3^{#1}}}}            % Derivada
\NC{\pd} [3][]{\Mc{\fr{\partial^{#1}#2} {\p #3^{#1}}}}            % Derivada parcial (de orden n en una misma variable)
\NC{\dm} [1]  {\Mc{\fr{\tx{D} #1}       {\tx{D} t}}}              % Derivada material
\NC{\dvr}[4][]{\Mc{\fr{\d^{#1} #2}      {\d #3^{#1}}\bigg|_{#4}}} % Derivada relativa
\NC{\pdr}[4][]{\Mc{\fr{\partial^{#1} #2}{\p #3^{#1}}\bigg|_{#4}}} % Derivada parcial relativa

% Derivadas small (tamaño pequeño):
\NC{\dvs}[3][]{\Mc{\frr{\d^{#1} #2}{\d #3^{#1}}}}                  % Derivada
\NC{\pds}[3][]{\Mc{\frr{\partial^{#1}#2}{\p #3^{#1}}}}             % Derivada parcial
\NC{\dms}[1]{\Mc{\frr{\tx{D} #1}{\tx{D} t}}}                       % Derivada material
\NC{\dvrs}[4][]{\Mc{\frr{\d^{#1} #2}{\d #3^{#1}}\Big|_{#4}}}       % Derivada relativa
\NC{\pdrs}[4][]{\Mc{\frr{\partial^{#1} #2}{\p #3^{#1}}\Big|_{#4}}} % Derivada parcial relativa

% Derivada parcial de orden superior:
\makeatletter
	\def\@IntergerSum{0}
	\def\@NonIntergerSum{}
	\NC{\@GetSumAux}[1]{
		\IfStrEq{#1}{}{}{\IfInteger{#1}{\edef\@IntergerSum{\fpeval{\@IntergerSum + #1}}}{\IfStrEq{\@NonIntergerSum}{}
		{\g@addto@macro\@NonIntergerSum{#1}}{\g@addto@macro\@NonIntergerSum{+ #1}}}}
	}
	\NC{\@GetSum}[5]{
		\gdef\@IntergerSum{0} \gdef\@NonIntergerSum{} \@GetSumAux{#2} \@GetSumAux{#3} \@GetSumAux{#4} \@GetSumAux{#5} \IfStrEq{\@NonIntergerSum}{}
		{\@GetExponent{#1}{\@IntergerSum}}{\IfStrEq{\@IntergerSum}{0}{\edef#1{^{\@NonIntergerSum}}}{\edef#1{^{\@NonIntergerSum + \@IntergerSum}}}}
	}
	\NC{\@GetExponent}[2]{
		\IfStrEq{#2}{}{\def#1{}}{\IfStrEq{#2}{1}{\def#1{}}{\def#1{^{#2}}}}
	}
	\NC{\@ShowVar}[2]{
		\IfStrEq{#1}{}{}{\partial#1#2}
	}
	\NC*{\@SumExponenet}{}
	\NC{\@PPD}[9]{
		\begingroup
			\@GetSum{\@SumExponenet}{#9}{#7}{#5}{#3} \@GetExponent{\@ExponenentV}{#3} \@GetExponent{\@ExponenentX}{#5} \@GetExponent{\@ExponenentY}{#7}
			\@GetExponent{\@ExponenentZ}{#9} \fr{\partial\@SumExponenet#1}{\@ShowVar{#2}{\@ExponenentV} \@ShowVar{#4}{\@ExponenentX} \@ShowVar{#6}{\@ExponenentY}
			\@ShowVar{#8}{\@ExponenentZ}}
		\endgroup
	}
	\NC{\@PPDs}[9]{
		\begingroup
			\@GetSum{\@SumExponenet}{#9}{#7}{#5}{#3} \@GetExponent{\@ExponenentV}{#3} \@GetExponent{\@ExponenentX}{#5} \@GetExponent{\@ExponenentY}{#7}
			\@GetExponent{\@ExponenentZ}{#9} \frr{\partial\@SumExponenet#1}{\@ShowVar{#2}{\@ExponenentV} \@ShowVar{#4}{\@ExponenentX} \@ShowVar{#6}{\@ExponenentY}
			\@ShowVar{#8}{\@ExponenentZ}}
		\endgroup
	}

	\NC{\ppd}  [5]{\@PPD{#1}{#4}{#5}{#2}{#3}{}{}{}{}}         % Derivada parcial en dos variables
	\NC{\pppd} [7]{\@PPD{#1}{#6}{#7}{#4}{#5}{#2}{#3}{}{}}     % Derivada parcial en tres variables
	\NC{\ppppd}[9]{\@PPD{#1}{#8}{#9}{#6}{#7}{#4}{#5}{#2}{#3}} % Derivada parcial en cuatro variables

	% Derivadas small (tamaño pequeño):
	\NC{\ppds}  [5]{\@PPDs{#1}{#4}{#5}{#2}{#3}{}{}{}{}}         % Derivada parcial en dos variables
	\NC{\pppds} [7]{\@PPDs{#1}{#6}{#7}{#4}{#5}{#2}{#3}{}{}}     % Derivada parcial en tres variables
	\NC{\ppppds}[9]{\@PPDs{#1}{#8}{#9}{#6}{#7}{#4}{#5}{#2}{#3}} % Derivada parcial en cuatro variables
\makeatother

%————————————————————————————————————————————————————————————————————————————————————————————————————————————————————————————————————————————

%============================================================================================================================================
% Comandos de matemática
%============================================================================================================================================

% Parte 7: Integrales.
%---------------------

\NC{\cte}{\tx{C}} % Constante

\NC{\Int} [4]{ \int\limits_{#1}^{#2} #3 \esp{5}\d#4}															% Integral en una  variable
\NC{\ii}  [7]{ \int\limits_{#3}^{#4}\esp{-3}\int\limits_{#1}^{#2} #5 \esp{5}\d#6\esp{3}\d#7}					% Integral en dos  variables
\NC{\iii} [7]{ \int\limits_{#5}^{#6}\esp{-3}\int\limits_{#3}^{#4}\esp{-3}\int\limits_{#1}^{#2} #7 \esp{5}\dxyz}	% Integral en tres variables
\NC{\Intj}[4]{ \int\limits_{#1}^{#2} #3 \esp{5}\dj#4}															% Integral de diferencial inexacto
\NC{\Oint}[4]{\oint\limits_{#1}^{#2} #3 \esp{5}\d #4}															% Integral ángulo sólido (cerrada)

% Comando auxiliar (para la integral de tres variables):
\NC{\dxyz}[3]{\d#1\esp{3}\d#2\esp{3}\d#3}

% Integrales de línea:
\NC{\ils}   [4][]{             \int\limits_{#2}^{#1} #3 \esp{5}\tx{d#4}} % Campo escalar
\NC{\ilos}  [4][]{            \oint\limits_{#2}^{#1} #3 \esp{5}\tx{d#4}} % Campo escalar (integral cerrada)
\NC{\iloscl}[4][]{   \ointclockwise\limits_{#2}^{#1} #3 \esp{5}\tx{d#4}} % Campo escalar, integral cerrada, orientación negativa (horario)
\NC{\iloscr}[4][]{\ointctrclockwise\limits_{#2}^{#1} #3 \esp{5}\tx{d#4}} % Campo escalar, integral cerrada, orientación positiva (anti-horario)

\NC{\ilv}   [4][]{             \int\limits_{#2}^{#1} #3 \por \d\vc{#4}} % Campo vectorial
\NC{\ilov}  [4][]{            \oint\limits_{#2}^{#1} #3 \por \d\vc{#4}} % Campo vectorial (integral cerrada)
\NC{\ilovcl}[4][]{   \ointclockwise\limits_{#2}^{#1} #3 \por \d\vc{#4}} % Campo vectorial, integral cerrada, orientación negativa (horario)
\NC{\ilovcr}[4][]{\ointctrclockwise\limits_{#2}^{#1} #3 \por \d\vc{#4}} % Campo vectorial, integral cerrada, orientación positiva (anti-horario)

% Integrales de superficie:
\NC{\iss} [4][]{ \iint\limits_{#2}^{#1} #3 \esp{5}\tx{d#4}}	% Campo escalar
\NC{\isos}[4][]{\oiint\limits_{#2}^{#1} #3 \esp{5}\tx{d#4}}	% Campo escalar (integral cerrada)
\NC{\isv} [4][]{ \iint\limits_{#2}^{#1} #3 \por \d\vc{#4}}	% Campo vectorial
\NC{\isov}[4][]{\oiint\limits_{#2}^{#1} #3 \por \d\vc{#4}}	% Campo vectorial (integral cerrada)

% Integrales de volumen:
\NC{\ivs} [4][]{\iiint\limits_{#2}^{#1} #3 \esp{5}\tx{d#4}}			% Campo escalar
\NC{\ivv} [4][]{\iiint\limits_{#2}^{#1} #3 \por \d\vc{#4}}			% Campo vectorial
\NC{\ivsr}[4][]{\iiint\limits_{#2}^{#1} #3 \esp{5}\tx{d}^3\vc{#4}}	% Notación vectorial


%============================================================================================================================================
% Comandos de matemática
%============================================================================================================================================

% Parte 8: Transformadas.
%------------------------

\NC{\TF}[3][]{\cal{F}_{#1\esp{3}\left[{#2}\right] \esp{6}\left({#3}\right)}}	% Transformada de fourier
\NC{\TL}[3][]{\cal{L}_{#1\esp{3}\left[{#2}\right] \esp{6}\left({#3}\right)}}	% Transformada de laplace

%============================================================================================================================================
% Comandos de matemática
%============================================================================================================================================

% Parte 9: Matrices.
%-------------------

\NC{\tras} [1]{{#1}^{\tx{t}}}	% Matriz transpuesta
\RC{\ast}  [1]{{#1}^*}			% Matriz transpuesta conjugada (hermítica)
\NC{\psinv}[1]{{#1}^+}			% Matriz pseudo-inversa
\Mo{\adj}{adj}					% Matriz adjunta
\Mo{\Nu}  {Nu}		% Núcleo
\Mo{\rang}{rang}	% Rango
\Mo{\tr}  {tr}		% Traza
\Mo{\diag}{diag}	% Diagonal

\NC{\comb}[2]{					% Número combinatorio (coeficiente binomial)
	\begin{pmatrix}
		#1 \\
		#2
	\end{pmatrix}
}
\NC{\lpm} {\begin{pmatrix}}		% Matriz con paréntesis
\NC{\rpm} {  \end{pmatrix}}
\NC{\lbm} {\begin{bmatrix}}		% Matriz con corchetes
\NC{\rbm} {  \end{bmatrix}}
\NC{\llam}{\begin{Bmatrix}}		% Matriz con llaves
\NC{\rlam}{  \end{Bmatrix}}
\NC{\lvm} {\begin{vmatrix}}		% Matriz con módulo (determinante)
\NC{\rvm} {  \end{vmatrix}}
\NC{\lvvm}{\begin{Vmatrix}}		% Matriz con norma
\NC{\rvvm}{  \end{Vmatrix}}

\NC{\dxd}{2 \Por 2} % Dimensiones
\NC{\txt}{3 \Por 3}
\NC{\nxn}{n \Por n}
\NC{\nxm}{n \Por m}

%============================================================================================================================================
% Comandos de matemática
%============================================================================================================================================

% Parte 10: Álgebra.
%———————————————————

% Operadores Matemáticos:
\NC{\dvd}{\esp{3}|\esp{3}}                                        % Divide a ...
\NC{\ndvd}{\esp{-3}\not\esp{-.5}|\esp{5}}                         % No divide a ...
\NC{\Div}[2][]{\Op{Div}_{#1}\esp{-3}\left\{{#2}\right\}}          % Conjunto de divisores
\NC{\Divcom}[3][]{\Op{DivCom}_{#1}\esp{-3}\left\{{#2,#3}\right\}} % Conjunto de divisores comunes
\NC{\congr} [3]{#1 \eqv #2\esp{20}\pr{\Op{mod}\esp{6}#3}}         % Congruente (módulo)
\NC{\ncongr}[3]{#1\neqv #2\esp{20}\pr{\Op{mod}\esp{6}#3}}         % No congruente (módulo)
\NC{\resto}[2][]{r_{#1\esp{3}\pr{#2}}}                            % Resto
\NC{\mcd}[2]{ \pr{#1:#2}}                                         % Máximo común divisor
\NC{\mcm}[2]{\cor{#1:#2}}                                         % Mínimo común múltiplo
\NC{\cop}{\perp}                                                  % Número coprimo
\NC{\fr}{\dfrac}                                                  % Fracción
\NC{\frr} [2]{\SB{0.7}{$\dfrac{#1}{#2}$}}                         % Fracción reducida
\NC{\frrr}[2]{\SB{0.6}{$\dfrac{#1}{#2}$}}                         % Fracción reducida
\NC{\inv}[1]{{#1}^{\tx{-1}}}                                      % Inversa
\NC{\rz}{\sqrt}                                                   % Raíz
\RC{\log}[2][]{\Op{log}_{#1}#2}                                   % Logaritmo

% Geometría:
\NC{\paral}{\parallel}                                        % Paralelo
\NC{\dist}[2]{\Op{dist}\left\{{#1\esp{2},\esp{-1}#2}\right\}} % Distancia entre dos puntos
\RC{\dim}[2]{\Op{dim}_{(\bb{#1})}#2}                          % Dimensión

%————————————————————————————————————————————————————————————————————————————————————————————————————————————————————————————————————————————

%============================================================================================================================================
% Comandos de matemática
%============================================================================================================================================

% Parte 11: Análisis complejo.
%-----------------------------

\NC{\conj}[1]{{#1}^*}										% Conjugado
\NC{\re}[2][]{\Mc{\Op{Re}^{#1}\esp{-3}\left\{{#2}\right\}}}	% Parte real
\NC{\im}[2][]{\Mc{\Op{Im}^{#1}\esp{-3}\left\{{#2}\right\}}} % Parte imaginaria
\Mo{\Arg}{Arg}												% Argumento
\Mo{\Log}{Log}												% Logaritmo principal
\NC{\res}[3][]{\Op{Res}^{#1}\left\{{#2,#3}\right\}}			% Residuo

%============================================================================================================================================
% Comandos de matemática
%============================================================================================================================================

% Parte 12: Análisis numérico.
%-----------------------------

\NC{\fl}[1]{\Op{fl}\left({#1}\right)}   % Punto flotante
\NC{\cond}[1][]{\Op{Cond}_{#1}}         % Número de condición

% Alfabeto griego:
\NC{\alfa}   {\alpha}            \NC{\Alfa}   {\tx{A}}
\NC{\vita}   {\beta}             \NC{\Vita}   {\tx{B}}
\NC{\gama}   {\gamma}            \NC{\Gama}   {\Gamma}
\NC{\del}    {\delta}            \NC{\Del}    {\Delta}
\NC{\eps}    {\varepsilon}       \NC{\Eps}    {\tx{E}}    \NC{\epsj} {\epsilon}
\NC{\zita}   {\zeta}             \NC{\Zita}   {\tx{Z}}
\NC{\ita}    {\eta}              \NC{\Ita}    {\tx{H}}
\NC{\tita}   {\theta}            \NC{\Tita}   {\Theta}    \NC{\titaj}{\vartheta}
\NC{\Iota}   {\tx{I}} 
\NC{\kapa}   {\kappa}            \NC{\Kapa}   {\tx{K}}
\NC{\lamda}  {\lambda}           \NC{\Lamda}  {\Lambda}
\NC{\mi}     {\mu}               \NC{\Mi}     {\tx{M}}
\RC{\ni}     {\nu}               \NC{\Ni}     {\tx{N}}
\NC{\omicron}{o}                 \NC{\Omicron}{\tx{O}}
\NC{\ro}     {\rho}              \NC{\Ro}     {\tx{P}}    \NC{\roj}  {\varrho}
\NC{\sigmaj} {\varsigma}
\NC{\taf}    {\tau}              \NC{\Taf}    {\tx{T}}
\NC{\yps}    {\upsilon}          \NC{\Yps}    {\Upsilon}
\let\phij\phi \RC{\phi}{\varphi}
\NC{\ji}     {\chi}              \NC{\Ji}     {\tx{X}}

%————————————————————————————————————————————————————————————————————————————————————————————————————————————————————————————————————————————