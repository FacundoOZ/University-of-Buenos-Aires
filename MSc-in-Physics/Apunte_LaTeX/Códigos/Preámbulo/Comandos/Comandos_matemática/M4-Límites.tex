
%============================================================================================================================================
% Comandos de matemática
%============================================================================================================================================

% Parte 4: Límites.
%——————————————————

\RC{\lim}  [3]{\Lim\limits_{#1\to#2} #3}                                             % Límite
\NC{\llim} [5]{\Lim\limits_{#3\to#4} \Lim\limits_{#1\to#2} #5}                       % Límite doble
\NC{\lllim}[7]{\Lim\limits_{#5\to#6} \Lim\limits_{#3\to#4} \Lim\limits_{#1\to#2} #7} % Límite triple

% Límites compactos:
\NC{\liim} [5]{\Lim\limits_{\left({#1,#2   }\right)\,\to\,\left({#3,#4   }\right)} #5}                                     % Límite doble
\NC{\liiim}[7]{\Lim\limits_{\left({#1,#2,#3}\right)\,\to\,\left({#4,#5,#6}\right)} #7}                                     % Límite triple
\NC{\limite}[4][]{\Mc{\Lim\limits_{\left({{#2}_1,\pors,{#2}_{#1}}\right)\,\to\,\left({{#3}_1,\pors,{#3}_{#1}}\right)} #4}} % Límite n-ésimo

\NC{\lims}[3]{\Lims\limits_{#1\to#2} #3}              % Límite superior
\NC{\limi}[3]{\Limi\limits_{#1\to#2} #3}              % Límite inferior
\RC{\max} [2][]{\Max\limits_{#1} \left\{{#2}\right\}} % Máximo
\RC{\min} [2][]{\Min\limits_{#1} \left\{{#2}\right\}} % Mínimo
\RC{\sup} [2][]{\Sup\limits_{#1} \left\{{#2}\right\}} % Supremo
\NC{\infm}[2][]{\Infm\limits_{#1}\left\{{#2}\right\}} % Ínfimo

% Operadores matemáticos auxiliares:
\Mo*{\Lim} {\tx{lím}}
\Mo*{\Lims}{\tx{lím}\,\tx{sup}}
\Mo*{\Limi}{\tx{lím}\,\tx{inf}}
\Mo*{\Max} {\tx{máx}}
\Mo*{\Min} {\tx{mín}}
\Mo*{\Sup} {sup}
\Mo*{\Infm}{\tx{ínf}}

%————————————————————————————————————————————————————————————————————————————————————————————————————————————————————————————————————————————