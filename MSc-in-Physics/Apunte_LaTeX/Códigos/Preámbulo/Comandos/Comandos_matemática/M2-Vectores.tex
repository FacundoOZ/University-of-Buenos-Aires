
%============================================================================================================================================
% Comandos de matemática
%============================================================================================================================================

% Parte 2: Vectores.
%-------------------

% Vectores:
\DeclareSymbolFont{mathbf}{OT1}{\familydefault}{\bfdefault}{n}
\DeclareSymbolFontAlphabet{\ff}{mathbf}
\Ms{\boldimath}{\mathord}{mathbf}{"10}
\Ms{\boldjmath}{\mathord}{mathbf}{"11}
\ExplSyntaxOn
	\NDC{\vc}{m}{											% Vector en negrita:
		\token_if_cs:NTF #1 {\bm{#1}}{						% si es una letra griega, aplicamos boldsymbol (\bm)
			\str_case:nnF{#1}{
				{i}{\ff{\boldimath}}						% si es i ó j, le quitamos el punto y lo hacemos bold,
				{j}{\ff{\boldjmath}}
			}
			{\ff{#1}}										% si no, es simplemente mathbf (\ff)
		}
	}
	\NDC{\vco}{m}{											% Vector en negrita, medido desde un centro de momentos (o):
		\token_if_cs:NTF #1 {{\bm{#1}}^{\SB{0.5}{(o)}}}{	% Ídem que vector común, pero le agrego (o) de exponente y achicado en 0.5
			\str_case:nnF{#1}{
				{i}{\ff{\boldimath}}
				{j}{\ff{\boldjmath}}
			}
			{{\ff{#1}}^{\SB{0.5}{(o)}}}
		}
	}
	\NDC{\vcm}{m}{											% Vector en negrita, medido desde el centro de masas (cm):
		\token_if_cs:NTF #1 {{\bm{#1}}^{\SB{0.5}{(\tx{cm})}}}{	% Ídem que \vco, pero con cm.
			\str_case:nnF{#1}{
				{i}{\ff{\boldimath}}
				{j}{\ff{\boldjmath}}
			}
			{{\ff{#1}}^{\SB{0.5}{(\tx{cm})}}}
		}
	}
	\NDC{\ver}{m}{											% Versor:
		\token_if_cs:NTF #1 {\bm{\hat{#1}}}{				% Es igual que \vc{}, pero lleva sombrero (\hat)
			\str_case:nnF{#1}{
				{i}{\ff{\hat{\boldimath}}}
				{j}{\ff{\hat{\boldjmath}}}
			}
			{\ff{\hat{#1}}}
		}
	}
\ExplSyntaxOff

% Tensor:
\NCX{\ten}[3][1=,3=]{\cal{#2}_{#1}^{#3}}

% Otro: vector en notación vieja
\RC{\Vec}{\overrightarrow}
\NC{\Veco}[1]{\Vec{{#1}^{\esp{-6}\SB{0.5}{(o)}}}}		% Vector medido desde un centro de momentos (o)
\NC{\Vecm}[1]{\Vec{{#1}^{\esp{-6}\SB{0.5}{(\tx{cm})}}}} % Vector medido desde el centro de masas (cm)

% Operaciones entre vectores:
\NC{\por}{\cdot} 											% Operador producto interno
\NC{\Por}{\times} 											% Operador producto vectorial
\NC{\Tor}{\otimes} 											% Operador producto tensorial
\NC{\PI}[2]{#1 \por #2}										% Producto interno   entre elementos
\NC{\PV}[2]{#1 \Por #2} 									% Producto vectorial entre elementos
\NC{\PT}[2]{#1 \Tor #2} 									% Producto tensorial entre elementos
\NC{\SPI}[2]{\left\langle{#1 \esp{3} , #2}\right\rangle}	% Espacio con producto interno
\NC{\compnt}[2]{\Op{comp}_{#1} \left(#2\right)}				% Componente
\NC{\proy}[2]{\Op{proy}_{#1} \left(#2\right)}				% Proyección ortogonal
\NC{\levi}[2][]{\epsj_{#2}^{#1}}							% Símbolo de Levi-Civita
\NC{\PVr}[6]{\left\langle{#2 #6 - #3 #5,#3 #4 - #1 #6,#1 #5 - #2 #4}\right\rangle}			% Producto vectorial resuelto
\NC{\PVc}[6]{\left\langle{#1,#2,#3}\right\rangle \Por \left\langle{#4,#5,#6}\right\rangle}	% Producto vectorial en componentes

% Operadores vectoriales:
\NC{\gr}[2][]{\nabla_{#1} {#2}}				% Gradiente
\RC{\div}[2][]{\nabla_{#1} \por \vc{#2}}	% Divergencia
\NC{\rot}[2][]{\nabla_{#1} \Por \vc{#2}}	% Rotacional
\NC{\lap}[2][]{\nabla_{#1}^2 {#2}}			% Laplaciano
\NC{\lapv}[2][]{\nabla_{#1}^2 \vc{#2}}		% Laplaciano vectorial
\NC{\blap}[1]{\nabla^4 {#1}}				% Bilaplaciano
\NC{\dal}[2][]{\Box_{#1} {#2}}				% D'Alambertiano
\NC{\dalv}[2][]{\Box_{#1} \vc{#2}}			% D'Alambertiano vectorial
\NC{\cnv}[2]{(\PI{\vc{#1}}{\nabla}) #2}		% Derivada Convectiva
\NC{\Rotr}[3]{								% Rotacional resuelto
	\left\langle{
		\pd{{#3}_{\xyz}}{y} - \pd{{#2}_{\xyz}}{z},
		\pd{{#1}_{\xyz}}{z} - \pd{{#3}_{\xyz}}{x},
		\pd{{#2}_{\xyz}}{x} - \pd{{#1}_{\xyz}}{y}
	}\right\rangle
}
