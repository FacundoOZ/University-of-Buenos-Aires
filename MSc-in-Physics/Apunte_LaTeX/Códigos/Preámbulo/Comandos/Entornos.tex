
%============================================================================================================================================
% Entornos
%============================================================================================================================================

% Figura (ancho opcional):
\newenvironment{fig}[3][0.5]{                       % Ancho por defecto = 0.5
    \begin{figure}[!htbp] \centering
        \includegraphics[width=#1\linewidth]{#2}
        \caption{#3}
}
{
    \end{figure}
}

% Ejemplo:
% \begin{fig}[0.8]{Imagen.png}{CAPTION.}
%    \label{fig:LABEL}
% \end{fig}

% Figura envolvente derecha (cantidad de líneas envolventes opcional):
\newenvironment{figr}[5][]{
    \wrapfigure[#1]{r}{#2\textwidth} \centering
        \includegraphics[width=#3\linewidth]{#4}
        \caption{#5}
}
{
    \endwrapfigure
}

% Figura envolvente izquierda (cantidad de líneas envolventes opcional):
\newenvironment{figl}[5][]{
    \wrapfigure[#1]{l}{#2\textwidth} \centering
        \includegraphics[width=#3\linewidth]{#4}
        \caption{#5}
}
{
    \endwrapfigure
}

% Ejemplo:
% \begin{figr}[20]{.5}{.9}{Imagen.png}{CAPTION.}
%   \label{figr:LABEL}
% \end{figr}

\ExplSyntaxOn
% Tabla centrada
    \NewDocumentEnvironment{tabc}{O{\rowcolor{blue!100}}mb}{\tabc:nnn {#1} {#2} {#3}}{}

    \seq_new:N \l__tabc_rows_seq
    \seq_new:N \l__tabc_firstrow_seq
    \seq_new:N \l__tabc_firstrow_bold_seq
    \tl_new:N \l__tabc_firstrow_tl
    \cs_new_protected:Nn \tabc:nnn
    {
        \seq_set_split:Nnn \l__tabc_rows_seq { \\ } {#3}                            % Separa las filas
        \seq_pop_right:NN \l__tabc_rows_seq \l_tmpa_tl                              % Busca un rastro \\
        \tl_if_blank:VF \l_tmpa_tl{\seq_put_right:NV \l__tabc_rows_seq \l_tmpa_tl}  % Si el último item no es nulo, lo agrega
        % Separa la primer fila
        \seq_pop_left:NN \l__tabc_rows_seq \l__tabc_firstrow_tl
        \seq_set_split:NnV \l__tabc_firstrow_seq { & } \l__tabc_firstrow_tl
        \seq_set_map:NNn \l__tabc_firstrow_bold_seq \l__tabc_firstrow_seq
        {\Tb{\textbf{##1}}}
        % Imprime la tabla
        \begin{tabular}{|*{#2}{c|}} \hline
            #1 \seq_use:Nn \l__tabc_firstrow_bold_seq { & } \\ \hline               % Primera fila
            \seq_use:Nn \l__tabc_rows_seq { \\ \hline } \\ \hline                   % El resto de filas
        \end{tabular}
    }

    % Tabla centrada a la izquierda
    \NewDocumentEnvironment{tabl}{O{\rowcolor{blue!100}}mb}{\tabl:nnn {#1} {#2} {#3}}{}

    \seq_new:N \l__tabl_rows_seq
    \seq_new:N \l__tabl_firstrow_seq
    \seq_new:N \l__tabl_firstrow_bold_seq
    \tl_new:N \l__tabl_firstrow_tl
    \cs_new_protected:Nn \tabl:nnn
    {
        \seq_set_split:Nnn \l__tabl_rows_seq { \\ } {#3}                            % Separa las filas
        \seq_pop_right:NN \l__tabl_rows_seq \l_tmpa_tl                              % Busca un rastro \\
        \tl_if_blank:VF \l_tmpa_tl{\seq_put_right:NV \l__tabl_rows_seq \l_tmpa_tl}  % Si el último item no es nulo, lo agrega
        % Separa la primer fila
        \seq_pop_left:NN \l__tabl_rows_seq \l__tabl_firstrow_tl
        \seq_set_split:NnV \l__tabl_firstrow_seq { & } \l__tabl_firstrow_tl
        \seq_set_map:NNn \l__tabl_firstrow_bold_seq \l__tabl_firstrow_seq
        {\Tb{\textbf{##1}}}
        % Imprime la tabla
        \begin{tabular}{|*{#2}{l|}} \hline
            #1 \seq_use:Nn \l__tabl_firstrow_bold_seq { & } \\ \hline               % Primera fila
            \seq_use:Nn \l__tabl_rows_seq { \\ \hline } \\ \hline                   % El resto de filas
        \end{tabular}
    }
\ExplSyntaxOff