
\title{\Huge\sc\SB{1.1}{\txsub{Licenciatura En Ciencias Físicas}}}							% Título.
\author{																					% Autor.
	\LARGE Facundo Otero Zappa | Viernes 10 de Julio del 2020 \vspace{2.5mm} \\ ${}$ \\
	\Large \cur{Departamento de Física, Facultad de Ciencias Exactas} \& \cur{Naturales, U.B.A.}
	}
\date{\vspace{2.5mm}\Large Contacto: \txcol{blue!100}{\txsub{facuotero20.88@outlook.com}}}	% Fecha.
\pagestyle{fancy} \fancyhf{} % Estilo:

\RC{\sectionmark}[1]{\markright{#1}{}}					% Muestra el número y nombre de la sección.
\RC{\chaptermark}[1]{\markboth{#1}{}}					% Muestra el número y nombre del capítulo.
\NC*\parttitle{} \let\origpart\part \RC*{\part}[2][]{	% Muestra El Número Y Nombre De La Parte.
	\ifx \\#1\\ \origpart{#2} \RC*\parttitle{#2}
	\else \origpart[#1]{#2} \RC*\parttitle{#1}
	\fi
}

\fancyhead[L]{Capítulo \thechapter \ --- \leftmark}		% Encabezado.
\fancyhead[C]{ }
\fancyhead[R]{Parte \thepart : \parttitle}
\lfoot{Sección \thesection \ --- \rightmark}			% Pie de página.
\cfoot{ }
\rfoot{Página \thepage}

\maketitle

\RC{\partname}{Parte}									% Nombre de títulos y subtítulos en español.
\RC{\chaptername}{Capítulo}
\RC{\abstractname}{Resumen}
\RC{\appendixpagename}{Apéndice}
\RC{\appendixname}{Apéndice}
\RC{\refname}{Bibliografía}
\RC{\figurename}{Figura}
\RC{\tablename}{Tabla}

\NC{\NL}{\newline} 										% Nuevo renglón.
\NC{\NS}{\noindent}										% Sin sangría.
\NC{\NP}{\newpage}										% Nueva página.

\cellspacetoplimit 2pt \cellspacebottomlimit 2pt		% Espaciado del contenido interno de las tablas.
\NC{\MC}{\multicolumn}									% Comandos de tablas.
\NC{\MR}{\multirow}
\NC{\HL}{\\ \hline}
\NC{\TNL}[2][Sl]{										% Nuevo renglón en tablas.
	\begin{tabular}[c]{@{}c@{}#1}
		#2
	\end{tabular}
}

\tableofcontents										% Índice.

\setupname
\setcounter{page}{1}									% Contador de páginas (del índice).